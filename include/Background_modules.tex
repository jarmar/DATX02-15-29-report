\section{Module systems}

As computer programs increase in complexity, a need for separation of concerns arises. Programming languages typically solve this problem by enabling the programmer to organize code into cohesive units known as modules, sometimes also referred to as packages or components. Besides the organizational benefits, there are technical merits to modular software as well. If a single module is changed, the entire project does not necessarily have to be recompiled; only the parts which depend on the module that was changed.

The concept of dependency is central in module systems, but details differ between programming languages. Some languages allow the programmer to define the interfaces between modules, for example by marking certain functions as \texttt{public} or giving a list of functions to export, while others automatically make all declarations available to other modules. Other features also differ between languages, such as allowing cyclical inter-module dependencies or whether dependencies between modules must be explicitly declared with \texttt{import} statements or similar.