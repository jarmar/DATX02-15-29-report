\section{Code quality and user interaction}

\subsection{Documentation}

A general consensus between group members to keep the code base well documented were not followed up or enforced by the group. This led to parts of the code being very well documented and others being less readable and understandable than preferred. This is something that could be fixed and should be if the Hopper language project is continued.

\subsection{Testing}

Plans for a thorough test suite were fulfilled and kept for a while but with the incremental changes made to the compiler pipeline maintaining the test code proved a challenge as tests became outdated. When a more stable pipeline is up this could once again be a priority.

\subsection{Usability}

\todo{better name?}

% Could be used as the pure part in an application. For example using Hopper to implement an difficult algorithm but using Erlang as the glue to the rest of the system. 

%% Something like: Furthermore, a compiler should be user friendly in the sense of giving descriptive error messages for faulty code, give warnings about inconvenient code, etc.