\section{Module system}
\label{sec:dai_modules}

\todo{Assigned Jakob}

\todo{subheadings good?}

The module system of Hopper is designed to be simple to implement.

Hierarchical module names are supported, e.g. \texttt{A.B.C}. Only a single source directory is supported, so there is a straight-forward translation from module names to file paths: the above example maps to \texttt{A/B/C.hpr}. This restriction removes the possibility of conflicts where multiple matching source files exist in different source directories.

\section{Dependency model}
Modules explicitly declare what functions and data types they export, and declare dependency on other modules using \texttt{import} statements. Importing a module means importing \emph{everything} that module exports, so fully qualified names are required when referring to imported functions or data types. For example, if module \texttt{A} defines a function \texttt{f} which module \texttt{B} wishes to use, module \texttt{B} must first \texttt{import A} and then use the notation \texttt{A.f} whenever it wants to use \texttt{f}. \todo{too cluttered?} This ensures that imported functions do not cause name clashes.

The import statements are used to build a \gls{dag} representing the dependency relations of the program. Verifying that the graph actually is acyclic ensures that no cyclical dependencies occur. Topologically sorting the nodes of the \gls{dag} gives a compilation order where the dependencies of each module are compiled before that module. This design ensures that type information for a module's dependencies is available at compile time, so that the compiler only has to consider a single module at a time.

\section{Interface files}
In order to propagate type information for a given module's exports into the modules that depend on it, the concept of interface files (similar in concept to those of Haskell's\cite{interfacefiles}, but much more basic) is used. They contain maps from identifiers to type information (which might have been given explicitly, or inferred by the type checker) for everything that is exported from a given module, and are written by the type checker after a successful run.

The actual implementation of the interface files simply uses derived \texttt{Show} and \texttt{Read} instances for the data types representing Hopper types.