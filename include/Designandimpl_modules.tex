\section{Module system}
\label{sec:dai_modules}

The module system of Hopper is designed to be simple to implement. Hierarchical module names are supported, e.g. \texttt{A.B.C}. Only a single source directory is supported, so there is a straightforward translation from module names to file paths. For instance, the above example maps to \texttt{A/B/C.hpr}. This restriction removes the possibility of conflicts where multiple matching source files exist in different source directories.

\subsection{Dependency model}

Modules explicitly declare what functions and data types they export, and declare dependency on other modules using \texttt{import} statements. Importing a module means importing everything that module exports, so fully qualified names are required when referring to imported functions or data types. For example, if module \texttt{A} defines a function \texttt{f} which module \texttt{B} wishes to use, module \texttt{B} must first \texttt{import A} and then use the notation \texttt{A.f} whenever it wants to use \texttt{f}. This ensures that imported functions do not cause name clashes.

The import statements are used to build a graph representing the dependency relations of the program. Verifying that the graph is a \gls{dag} ensures that no cyclical dependencies occur. \glslink{topsort}{Topologically sorting} the nodes of the \gls{dag} gives a compilation order where the dependencies of each module are compiled before that module. This design ensures that type information for a module's imports is available at compile time, so that the compiler only has to consider a single module at a time. For example, with the module hierarchy in figure~\ref{fig:depgraph}, attempting to compile module \texttt{A} yields a compilation order of modules \texttt{D}, \texttt{B}, \texttt{E}, \texttt{C}, \texttt{A}.

\begin{figure}[h!]
\centering
  \includegraphics[width=0.6\pdfpagewidth]{figure/depgraph}
  \caption{Example dependency graph}
  \label{fig:depgraph}
\end{figure}

\subsection{Interface files}

In order to propagate type information for a given module's exports into the modules that depend on it, the concept of interface files is used. Hopper's interface files are similar in concept to those of Haskell's\cite{interfacefiles}, but much more basic. They contain a text representation of maps from identifiers to type information (which might have been given explicitly or inferred by the type checker) for everything that is exported from a given module.
