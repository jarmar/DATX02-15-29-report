\chapter{Discussion}

\todo{Assigned Johan L}
\todo{Expand on the results, why did some stuff not make it etc}
\todo{Follow structure of table of contents (Method/Implementation/Results)}

\begin{itemize}

  \item \todo{Differences between report plan and finished product}
  \begin{description}
    \item[Starting with a vision]
      The first task the group tackled was designing the new language. Initially the discussion revolved around what features from the source languages would be used and how to make the language not only useful but user friendly as well. Aiming high the group agreed on a goal that felt realistic. 
    \item[Managing scope]
      Combining the extent of features planned for the language and the groups previous inexperience of writing compilers and languages the group needed to manage the scope of the project. Setting the goal at a working prototype made clear the groups intention of not delivering a finished and polished product. To get the work going the group also decided to limit the features included in a first version. This subset was seen as simpler and would allow the group to get a full pipeline working which could then be built upon.
    \item[Parts lagging]
      The initial weeks saw some parts of the pipeline making fast progress and others barely none. What was initially planned as a small subset to be implemented did not lend itself fully to the type checking. The type checking suffered not only from being a heavy theoretical subject with great depth but also from not lending itself to easily be done in small steps. Other parts of the pipeline suffered from not having a functional type checker. New features were slowly added to the project but without a clear idea of what was needed all work was slowed considerably. Initial plans of having group members switch between compiler phases became harder as the weeks passed since the group members responsible for the type checker needed to finish their part.
    \item[Report plan]
      Come the report plan the group was still hopeful of reaching their goal. The setbacks felt minor and as long as a working pipeline was up new important features could be added step by step.
    \item[Half time]
    \item[Finished product]
  \end{description}

  \item \todo{Difficulties make typechecker}
  \begin{description}
    \item[Lots of theory]
      Type theory is a subject of both great breadth and depth. Doing research the responsible group members had to scan a large number of resources for information in a field in which they were not yet fully immersed. This led to very little implementation being done early on and a difficulty assessing the amount of work that would be needed going forward. It also meant that inviting other group members to help became difficult since the prerequisite knowledge gathered in the first few weeks were not easily transferred. A few weeks in to the project half of the group members were working on type checking but no code had yet been produced.
    \item["Too many chefs"/communication]
      Not only was communication regarding type checking difficult within the group but also between the group members responsible for it. The field was hard to grasp and decisions of what path to take differed between the members seeing the dawn of two separate type checking branches in the repository. A type system had been chosen and the implementations were worked with the hope of ideas converging later.
    \item[Branching, no merging]
  \end{description}

  \item \todo{AST changes, transformations, simplifications}
  \item \todo{"well documented" code, possible future additions}

  \begin{description}
    \item[Massaging for the type checker/code gen]
    \item[Type checker problems spill over]
  \end{description}
  \item \todo{Where to hack into erlc, why core erlang}

  \item \todo{What features did we want but didnt have time to implement?}
\end{itemize}
