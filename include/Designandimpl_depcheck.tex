\section{Dependency checker}
\label{sec:dai_depcheck}

\todo{Assigned Jakob}
\todo{Consider breaking up into subheadings, esp. for interface files}

The dependency model of Hopper is rather simple. Modules explicitly declare what functions and data types they export, and declare dependency
on other modules using \texttt{import} statements. Hierarchical module
names are supported, e.g. \texttt{A.B.C}. Only a single
source directory is supported, so there is a straight-forward map from module
names to file paths: the above example maps to \texttt{A/B/C.hpr}. This
restriction simplifies the work of the dependency checker, by removing the concept of conflicts where multiple matching source files exist in different source directories.

The import statements are used to build a \gls{dag}. Verifying that the graph actually is acyclic ensures that no cyclical dependencies occur. Topologically sorting the nodes of the \gls{dag} gives a compilation order where the dependencies of each module are compiled before that module. This design ensures that type information for a module's dependencies is available at compile time, so that the compiler only has to consider a single module at a time.

In order to propagate type information for a given module's exports into the modules that depend on it, the concept of interface files (similar in concept to those of Haskell's\cite{interfacefiles}, but much more basic) is used. They contain type information (which might have been given explicitly, or inferred by the type checker) for everything that is exported from a given module. The actual implementation of the interface files simply uses derived \texttt{Show} and \texttt{Read} instances for the data types representing Hopper types.

\todo{further explain why this is a good organisation}

