\chapter{Background}

\todo{Assigned Johan L}

% WHAT IS A COMPILER 

The Erlang virtual machine is appreciated for its excellent support for
concurrency, distribution and error handling, allowing soft-real-time systems
with high uptimes - famously demonstrated by the 99.9999999\% uptime of
Ericsson’s AXD301 telecom switch from 1998. The features of the BEAM (Erlang’s
main virtual machine implementation) aren't tied to Erlang itself: one can
enjoy the benefits of the runtime with a different syntax and semantics.
Previous efforts in this field include: Elixir, a language with Ruby-like
syntax; Javascript flavoured Erlang, Lisp flavoured Erlang and a YHC Haskell
to Erlang transpiler that preserves laziness.

Attempts to port Erlang features to Haskell have also been made, illustrating
the demand for a combination of the languages’ features. Cloud Haskell is
‘Erlang in a library’, but differences in the Erlang and Haskell runtime
system make shipping expressions between nodes far more verbose than in Erlang.

Haskell is known for its terse and elegant syntax, in addition to a powerful
type system that prevents many programming errors and makes it easier to
reason about code. Erlang developers already use code analysis tools to write
typed code, but these tools don’t interact well with message passing where the type received can be hard to predict. Combining the compile-time features of Haskell with the run-time benefits of Erlang into a language interoperable with the latter would be a great boon for Erlang developers.

%To take advantage of both BEAM's great concurrency model with the safety of 
%Haskell's typesystem this project is intended to design and implement a new 
%language named Hopper.  

New languages are often developed to allow for a higher level of abstraction and a higher level of reasoning about algorithms. To be able to run these new languages compilers are built to convert the high level code down to machine code, to run directly on hardware, or to a lower level language to make use of that language's compiler. 

% GENERAL PIPELINE FIGURE

\todo{Update figure: Typechecker -> Type checker}

\begin{figure}[h!]
  \centering
  \includegraphics[width=0.6\pdfpagewidth]{figure/general-pipeline}
  \caption{A general compiler pipeline}
  \label{fig:generalpipeline}
\end{figure}

% HOW IS A COMPILER BUILT 

The structure of a compiler can be likened to a pipe where text flows through. 
The text flowing through the pipe is transformed between different representations and is in
the end output as runnable code to either run on the computer itself or on a virtual machine
that emulates a computer. 

In figure ~\ref{fig:generalpipeline} you can see the flow handling the source code.
It should be noted that the type checker is optional for a language, it can
fully rely on the grammar to expose errors. This will lead to a more expressive, 
or less restricted, language but increases the risk of runtime crashes due to type errors.

We start by describing the first step of a compiler, the lexer and parser.

%Now is Hopper really a transpiler or a translating compiler. Instead of 
%outputting runnable code it outputs other code that is interpreted and compiled
%by the Erlang compiler, erlc. With this the project could very much be 
%simplified (ref to read more in discussion).


% OTHER SECTIONS

\section{Parsing source code} \label{sec:bnfc}
Similar to a normal language a programming language is built from a grammar.
The grammar describes ways in which the smaller pieces of the language can be put
together to form bigger ones, often with more meaning.
From this grammar a lexer is produced. The lexers job is to match all words in the
source code to tokens in the grammar. For example in english the word train 
would match a noun token. 

The next step is to combine all these tokens to expressions or sentences. This
is done with a parser. To make the language as expressive as possible you build
up bigger expressions out of smaller expressions. This will create a tree
structure that will be easier to work with. 

To save a lot of work the lexer and parser can be automatically generated from
a tool when the grammar is written in Backus-Naur-form or BNF. This BNF converter, or
BNFC for short, is developed by Chalmers and used in earlier courses meaning that
the group members was already famililar with the tool (plt ref here).



\section{Renaming and simplifying}

\todo{Assigned David}

While BNFC is a great tool for creating a parse tree from raw text the resulting tree structure is verbose and has a lot of constructors. To simplify this representation a renamer step is introduced in the pipeline. This step converts the parse tree to our minimally designed \acrlong{ast} or \acrshort{ast} for short. 

\begin{figure}
\begin{lstlisting} 
f = if a                      f = case a of
      then b        =>              True  -> b
      else c                        False -> c
\end{lstlisting}
\caption{Simple transformation to a more general form}
\label{lst:renamer1}
\end{figure}

In this step a few grammar rules could be translated to more general expressions to simplify the AST and reduce the number of cases the typechecker need to cover, see figure \ref{lst:renamer1}.

The Renamer has a few more task to take care of. First it gathers functions defined with patternmatching and merge them. Second it annotates the functions with the module name. This removes the ambiguity betweeen functions with the same name but defined in different modules. Lastly it checks a list of \acrlong{bif}s or \acrshort{bif}s for short to switch out them to calls to the virtual machine instead of regular user defined function calls.

\todo{BIFs, Collect definitions, qualify}

\section{Dependency checking}

\todo{Assigned Jakob}

\todo{motivation}
\section{Type checking}

\todo{POSSIBLY RENAME TO TYPE CHECKING, but mention that it is a kind of semantic analysis. Don't worry I will!//Liam}
\todo{Assigned Liam}

Programmers inevitably make mistakes. No person can be vigilant towards every possible slip-up all the time. While there are many tasks computers excel at, understanding the programmer’s intent and detecting differences between it and the behaviour of the programs they write (i.e. bugs) is not one of them. To aid in the automatic detection of bugs, so called type systems have been devised.

Type systems are rules governing what is and is not a valid program; they group values in the programming language into types, and specify rules about what is or is not a valid interaction between them. For example, in most if not all languages it is not acceptable to apply a number to a value as if it were a function. The difference between dynamically and statically typed programming languages is whether the erroneous application is discovered at run time or compile time.

Type systems automatically detect nonsensical code at compile time – (such as the program treating a Boolean as a list of Booleans) which reveals many bugs shortly after they are written, allowing the programmer to diagnose bugs when the code they’ve written is fresh in their mind. Types also allow programmers to communicate properties of their program to others in a concise way. For example: in Haskell a program which has some arbitrary interaction with the outside world and returns a Boolean to the caller when it’s done can be distinguished by its type (IO Bool) from one which simply calculates the truth value of a pure mathematical proposition (which would be of type Bool).

Hopper uses a type system similar to Haskell's to provide compile-time guarantees about program behaviour on Erlang's BEAM virtual machine. BEAM is focused on supporting concurrent processes and message passing of dynamically typed data between them (a side effect-rich environment with ample opportunities for type errors). In section 4.5, we will demonstrate the advantages of using types to describe concurrent programs.


%An important design decision to make when implementing a programming language is whether to perform some semantic analyses of the code at compile time and what to include in such case. Semantic analyses can for example contribute error detection and optimized code. The cost is increased complexity of the compiler and for some kinds of analysis prohibitive compilation times.

%A common and well studied category of analyses is type inference or type checking. Type systems vary in their expressiveness but essentially allow us to prove desireable properties of our code.

\todo{Motivation of Hopper is to bring type checking to erlang}

\todo{What we want from the type system}

\todo{move 2.4.1 Type inference to design and impl}

\todo{Choice of type system, if this is kept remember to keep it about different type systems and what the choice implies, not the choice of HM}

\todo{Constraint generation by means of inference rules}

\todo{Constraint solving}

\todo{Reporting type results}

%------------------

%what
%  in general
%    a step in the compilation process
%    examine details of the code which cannot be captured be the grammar itself.
%    example control flow analysis
%  type inference
%    a formal system
%    system F in Haskell
%    we use Hindley-Milner type system to begin with

%why
%  in general
%    prove properties
%    show correctness/show absence of errors
%  type inference
%    tractable
%    well established

%how
%  preprocess
%    the simple language
%      lambda calc
%        expressiveness etc
%        extensions (let, fix)
%    transformations
%      parse -> AST ?
%      AST -> simple language
%      important: keep semantics
%  type inference
%    generating constraints by inference rules
%    solving constraints by solving rules
%  postprocess
%    looking upp the solved types and reporting them

\section{Core Erlang and BEAM}

Core Erlang \cite{CoreErlangIntro} is an intermediate language in the Erlang compilation suite, mainly used for simplifiying operations on Erlang source code. Such operations can include (but not limited to) parsing, compiling and debugging. Some of the main purposes of Core Erlang are to be as regular as possible and to provide clear and simple semantics. The language is since release 8, released in 2001, an official part of the OTP/Open Source Erlang Distribution.

BEAM is the name of the Erlang virtual machine. Erlang and Core Erlang source code may be compiled to .beam files, which contain instructions from the BEAM instruction set. These instructions are interpreted by the virtual machine at run time.
