% CREATED BY DAVID FRISK, 2014
The Hopper language\\
A Haskell-like language on the Erlang VM\\
DATX02-15-29\\
Department of Computer Science and Engineering\\
Chalmers University of Technology\\
University of Gothenburg\\

\thispagestyle{plain}			% Supress header 
\section*{Abstract}

\todo{Assigned to Johan L}

\todo{More tips from link:}

http://www.th-wildau.de/fileadmin/dokumente/studiengaenge/europaeisches_management/dokumente/Dokumente_EM_Ba/Abstracts_in_English.pdf

\todo{FINISH ABSTRACT AFTER REST OF THESIS IS DONE}

\todo{Add actual results/conclusions etc}

The primary goal of this thesis is the design and implementation of a Haskell-like language for the Erlang VM. The name of this new language is Hopper.
In the background we follow the steps needed to turn high level Hopper code into code runnable on the virtual machine. There is a presentation of the different parts of a compiler explaining the need and use of each. The thesis then states what methods was used when developing the language. An agile development with Git as revision control system and Haskell as implementation language for the compiler is described. Design and implementation is then covered, once again going through the compiler pipeline, further explaining what choices were made in implementation and how it affected the compiler.
The results/conclusions will be summarized here...




%This report aims to describe the design and implementation of a Haskell like language for the Erlang VM.

\todo{Describe why this is interesting}

\todo{Potential glossary words: VM, compiler}

%The language will feature a Haskell-like syntax, offer a typesystem
%similar to that of Haskell and expose, in a safe way, the powerful features of
%the Virtual Machine. To utilize the big Erlang ecosystem it will also integrate
%with current OTP and Erlang libraries.


\section*{Sammanfattning}
Översätt när vi har bestämt oss för ett bra abstract...


% KEYWORDS (MAXIMUM 10 WORDS)
%\vfill
%Keywords: Haskell, Erlang, Erlang Core, BEAM.

\newpage				% Create empty back of side
\thispagestyle{empty}
\mbox{}