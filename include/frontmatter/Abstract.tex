% CREATED BY DAVID FRISK, 2014
The Hopper language\\
A Haskell-like language on the Erlang VM\\
DATX02-15-29\\
Department of Computer Science and Engineering\\
Chalmers University of Technology\\
University of Gothenburg\\

\thispagestyle{plain}			% Supress header

\setlength{\parindent}{15pt}

\section*{Abstract}

%Abstract: Provide a short summary of your study, including background + goal (1-3 sentences), main findings (1-5 sentences), and main conclusions (1-3 sentences)}

%Erbjuder ett snabbt sätt för läsaren att avgöra om rapporten är relevant för vederbörande och tillför viktig kunskap / viktiga resultat / metoder. Sammanfattningen ska ange laborationens / rapportens syftes- eller problemformulering, resultat, slutsatser samt metoder eller teori i mån av utrymme och relevans.

% 150-250 words

% Stycke 1: Intro: thesis aim, project goal
%The following report aims to ...
%We will discuss ... and whether it appears to be feasible

The following report aims to give insight into the design and implementation of a statically typed functional language for the Erlang virtual machine, discussing how such an implementation may be approached and whether it appears to be feasible. The primary goal of the project was to design a grammar specification and implement a compiler for such a language.

% Stycke 2: "Rapporten beksriver vad som har implementerats"

Over the course of the project a prototype language and a compiler for that language were developed. The project followed an iterative development process with Scrum as a basis. Notable modules of the compiler are the parser generated from a \acrshort{bnf} grammar, the type checker implementing a \gls{hmtypesystem} and the code generator generating \gls{core} source code.

% 

% Stycke 3: "Rapporten diskuterar resultaten och drar slutsatser"

The result of the project is Hopper, a basic functional programming language with an accompanying compiler, featuring polymorphic \glspl{adt}, pattern matching and lambdas. The language also has a module system and some integration with Erlang. 

In conclusion, the project was largely successful in its mission to create a typed functional language on the Erlang VM and has the potential to be developed further.

\section*{Sammanfattning}

\begin{otherlanguage}{swedish}
Den följande rapporten syftar till att ge insikt i utformningen och implementeringen av ett statiskt typat funktionellt programmeringsspråk för Erlangs virtuella maskin, och diskuterar hur man kan gå tillväga med en sådan implementation samt om det verkar genomförbart.

Under projektets gång har ett prototypspråk samt en kompilator för språket utvecklats. Projektet följde en iterativ utvecklingsprocess med Scrum som utgångspunkt. Noterbara delar av kompilatorn är parsern, genererad utifrån en \acrshort{bnf}-grammatik, typkontrolleraren som implementerar ett \glslink{hmtypesystem}{Hindley-Milner-typsystem} samt kodgeneratorn som genererar \gls{core}-källkod.

Resultatet av projektet är Hopper, ett enkelt funktionellt programmeringsspråk med tillhörande kompilator, med polyformiska \glslink{adt}{algebraiska datatyper}, mönstermatchning och lambdauttryck. Språket har även ett modulsystem och viss integration med Erlang.

Sammanfattningsvis var projektet till största delen lyckat i sin uppgift att skapa ett typat funktionellt språk på Erlangs virtuella maskin och har potential för vidareutveckling.
\end{otherlanguage}

% KEYWORDS (MAXIMUM 10 WORDS)
%\vfill
%Keywords: Haskell, Erlang, Erlang Core, BEAM.

\newpage				% Create empty back of side
\thispagestyle{empty}
\mbox{}