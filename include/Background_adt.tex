\section{Algebraic data types}

\Glspl{adt} are types composed by other types, and are often used to define complex data structures. The values of \glspl{adt} are created using constructors. Constructors take zero or more arguments and return a composite value containing the given values in the defined structure. Such a composition may then be deconstructed and analyzed, typically using pattern matching. An \gls{adt} may have several constructors with different combinations of data, and may also be recursive. This provides a simple but powerful way of expressing rich data structures.

\Glspl{adt} do not exist in Erlang in the same way as they do in Haskell, i.e. with their heavy reliance on constructors. When using pattern matching in Erlang, an entity called atom is instead used for handling specific cases. An atom is a literal constant with a name. Variables may hold atoms, and said variables can be pattern matched over different atoms for identifying different cases.