\section{History}

The Erlang virtual machine is appreciated for its excellent support for
concurrency, distribution and error handling, allowing soft real-time systems
with high uptimes --- famously demonstrated by the 99.9999999\% uptime of
Ericsson’s AXD301 telecom switch from 1998\cite{ninenines}. The features of the BEAM (Erlang’s
main virtual machine implementation) aren't tied to Erlang itself: one can
enjoy the benefits of the runtime with a different syntax and semantics.
Previous efforts in this field include: Elixir\cite{elixir}, a language with Ruby-like
syntax; Lisp flavoured Erlang\cite{lfe} and YHC\cite{yhc} a Haskell
to Erlang transpiler that preserves laziness.

Attempts to port Erlang features to Haskell have also been made, illustrating
the demand for a combination of the languages’ features. Cloud Haskell\cite{cloudhaskell} is
‘Erlang in a library’, but differences in the Erlang and Haskell runtime
system make shipping expressions between nodes far more verbose than in Erlang.

Haskell is known for its terse and elegant syntax, in addition to a powerful
type system that prevents many programming errors and makes it easier to
reason about code. Erlang developers already use code analysis tools to write
typed code, but these tools don’t interact well with message passing where the type received can be hard to predict. Combining the compile-time features of Haskell with the run-time benefits of Erlang into a language interoperable with the latter would be a great boon for Erlang developers.

%To take advantage of both BEAM's great concurrency model with the safety of 
%Haskell's typesystem this project is intended to design and implement a new 
%language named Hopper.