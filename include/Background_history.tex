

\section{History}

The Erlang virtual machine is appreciated for its excellent support for concurrency, distribution and error handling, allowing soft real-time systems with high uptimes, famously demonstrated by the 99.9999999\% uptime of Ericsson’s AXD301 telecommunication switch from 1998 \cite{ninenines}.

Haskell is known for its terse syntax, in addition to an expressive type system that prevents many programming errors and makes it easier to reason about code. Erlang developers already use code analysis tools to write typed code, but these tools don’t interact well with message passing where the type received can be hard to predict.

The features of the \gls{beam}, \glsdesc{beam}, are not tied to Erlang itself. \Gls{beam} hosts a variety of languages, so it is possible to enjoy the benefits of the runtime system with a different syntax and semantics. Previous efforts in creating alternatives to Erlang as a front-end language include Elixir\cite{elixir}, a language with Ruby-like syntax, and Lisp flavored Erlang \cite{lfe}.
