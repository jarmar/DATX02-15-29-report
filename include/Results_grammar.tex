\section{The Grammar}

The final grammar for the language supports most of the necessary syntax for a pure functional programming language.  The full grammar can be found in appendix \ref{app:grammar}. The main supported features are:

% * Indent sensitive
% * Type signatures 
%   * Parameterized types
% * ADTs
% * Pattern matching
%   * Tuples
%   * Constructors
%   * Variables
%   * Primitives
%   * Wildcard/ignore
% * Case expressions
% * Lambda expressions
% * If expresssions
% * infix Operators
% * primitives
% * Tuples
% * Qualified functions and constructors
\begin{itemize}
  \item Indentation sensitivity
  \item Function definitions and type signatures
  \item Polymorphic abstract data types
  \item Lambda and case expressions
  \item Primitive operators
  \item Pattern matching of primitives, tuples, constructors and variables
  \item Module definitions with imports and exports
  \item Qualification of functions, constructions and type constructors
\end{itemize}

Additional grammatical constructs were planned and designed but not implemented. Most of those features were various kinds of syntactic sugar, for example lists and tuples, but also syntax for simplifying repetitive tasks such as \texttt{IO} sequencing.

Some primitive operators, such as arithmetic and relative, were hard coded into the product. There were plans for implementing a more generic solution for this where users would be able to import any Erlang \glspl{bif} and functions as desired in a type safe manner.
