\section{The Grammar}

The final grammar for the language supports most of the necessary syntax for a pure functional programming language.  The full grammar is in appendix \ref{app:grammar}, but the main features are:

% * Indent sensitive
% * Type signatures 
%   * Parameterized types
% * ADTs
% * Pattern matching
%   * Tuples
%   * Constructors
%   * Variables
%   * Primitives
%   * Wildcard/ignore
% * Case expressions
% * Lambda expressions
% * If expresssions
% * infix Operators
% * primitives
% * Tuples
% * Qualified functions and constructors
\begin{itemize}
  \item Indent sensitive
  \item Function definitions and type signatures
  \item Polymorphic abstract data types
  \item Lambda and case expressions
  \item Operators
  \item Pattern matching of primitives, tuples, constructors and variables
  \item Module definitions with imports and exports
  \item Qualification of functions, constructions and type constructors
\end{itemize}

Additional grammatical features % other word for feature
were planned and designed but not implemented. Most of those features were syntactic sugar (e.g. for lists and tuples), but also syntax for simplifying repetitive tasks such as \texttt{IO} sequencing, also known in Haskell as \texttt{do}-notation. 