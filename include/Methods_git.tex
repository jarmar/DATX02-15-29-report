\section{Code revision control}

\todo{Assigned Johan L}

\todo{past or present tense?}

When developing the Hopper language, and implementing its compiler, Git was used for code revision control.
Revision control systems enable development using several simultaneous contributors.
A central repository is used to host the latest version of the code with group members
being able to add features locally and push any changes to the central repository when done.  
Furthermore Git, as most revision control systems, offers branching. 

\subsection{Branching}

Branching is a method of realizing a non-linear development history and can be used for 
testing different solutions in parallel without disturbing other contributors work.
This means that the group can have several versions of the code and always have a
latest working version while simultaneously working on different new features
to be added. Merging, putting the different versions of the code together, is often
quite simple when using Git and the overall work flow is improved compared to
multiple users working on the same piece of code. 
Due to the project's experimental nature, Git was the chosen revision control system 
for its branching functionality.


\todo{reference Git}
