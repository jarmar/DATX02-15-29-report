\section{Type checking}

Programmers inevitably make mistakes. While there are many tasks computers excel at, understanding the programmer's intent and detecting differences between it and the actual behavior of the programs they write is not one of them. To aid in the automatic detection of errors, \gls{seman} may be used. One such kind of analysis is type checking, which is enabled by having a type system.

\Glspl{typesystem} are formal systems which enable proof of absence of certain kinds of errors. They are typically defined in terms of inference rules.

Type systems can be used to help in rejecting invalid programs. They group values in the programming language into types, and specify rules about what is or what is not a valid interaction between them. For example, a type system might be defined to reject a program that tries to apply a number to a value as if it were a function.

There are two main kinds of type systems: dynamic and static. The most noticeable difference between dynamically and statically typed programming languages is whether type errors are discovered at run time or compile time.

Static type systems automatically detect type incorrect code at compile time, such as a program treating a Boolean as a list of Booleans. This allows the programmer to diagnose errors during the implementation phase. In contrast, dynamic type systems evaluate types during runtime, allowing the programmer to manipulate data in a more free manner.

An additional benefit of type systems is that they can be used as a means of specifying and documenting properties of a program in a concise way. Typically, a function may be annotated with a type signature which states the types of its arguments and its return type. 
\todo{Check this part again}

%For example, in Haskell a program which might have some side effects and returns a Boolean can be distinguished by its type (\texttt{IO Bool}) from a function which simply calculates a value of a pure mathematical proposition, which would be of type \texttt{Bool}.


% old liam stuff
% \section{Type checking} %Snälla konsultera mig innan stora ändringar av innehållet görs i Type checking-delar! %Åtminstone kommentera bara ut det som tas bort. %a) att det finns olika sorters semantisk analys påpekas i Compiler-delen %b) error/= exakt bug, dynamiska typsystem evaluerar inte typer att runtime etc. Programmers inevitably make mistakes. While there are many tasks computers excel at, understanding the programmer's intent and detecting differences between it and the actual behavior of the programs they write is not one of them. %To aid in the automatic detection of errors, semantic analysis may be used. One such kind of analysis is type checking, which is enabled by having a type system. To aid in the automatic detection of bugs, type systems have been devised. Type systems are formal rules governing what is and is not a valid program. They group values into types and specify what is and is not a valid interaction between those values. When a program type checks, the programmer can be assured that a common class of error is absent. For example, in most if not all languages it is not acceptable to use a number as if it were a function. The difference between dynamically typed languages such as Erlang and statically typed languages such as Haskell is whether the erroneous application is discovered at run time or compile time. An additional benefit of type systems is that they can be used as a means of specifying and documenting properties of a program in a concise way. For example, a program that sends and receives messages or has other side effects can be given a different type than one which simply calculates a value. In Section~\ref{sec:dai_tc}, we will demonstrate the advantages of using types to describe the kind of concurrent programs run on the BEAM virtual machine. %Type systems are formal systems which enable proof of absence of certain kinds of errors. They are typically defined in terms of inference rules. <<< Bra, ska se om jag kan arbeta in det här så att texten har flow. %Type systems can be used to help in rejecting invalid programs. %They group values in the programming language into types, and specify rules about what is or what is not a valid interaction between them. For example, a type system might be defined to reject a program that tries to apply a number to a value as if it were a function. %There are two main kinds of type systems: dynamic and static. The most noticeable difference between dynamically and statically typed programming languages is whether type errors are discovered at run time or compile time. \todo{Check this part again} %For example, in Haskell a program which might have some side effects and returns a Boolean can be distinguished by its type (\texttt{IO Bool}) from a function which simply calculates a value of a pure mathematical proposition, which would be of type \texttt{Bool}.
