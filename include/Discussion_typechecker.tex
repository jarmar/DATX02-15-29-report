\section{Type checker}

The type checker is a success in the sense that it implements the agreed-upon feature of the language's type system without any known bugs. However, significant technical debt was accumulated during the development process. Since the type checker was originally implemented using a different set of data types to describe programs, a quick-and-dirty conversion module was written when it was time to integrate it with the main development branch. That module has never been removed; instead, both sets of data types and the conversion module have been maintained. Much of the type checker was written during a 'Eureka moment' over a few days late in the first period. The code remains inconsistently commented. More focus has been put on fixing bugs and adding features rather than cleaning up the module as a whole; the end result is a codebase that is barely readable by its author, let alone other team members.

There is not much to say about design decisions in implementing the Hindley-Milner type system; there weren't any. We simply followed the recipe.
%Ha med denna sektion? Känns som förra slutar rätt så abrupt.
That said, the type system contains a number of other interesting ideas. In a production version bif access function Prim.apply will almost certainly be implemented in some form, in addition to typesafe and non-typesafe receive clauses. Hopper's (experimental) version of ad-hoc polymorphism is less powerful than Haskell's and would likely be replaced.
%Story time! Passar det här? Det är en förklaring hur grejs gick till.
%Också, borde vi identifieras här?
%Initially, three out of six members of the team were assigned to the type checker. Partly due to disagreements over the structure of core data structures of Hopper the three-man task force was divided into a group of two and a group of one, each working on a different version of the type checker. %Is that even a group?
%For a period of time, it looked increasingly likely that the two-man team would "win out": the one-man team was swamped by work in another course (Advanced Functional Programming [course code])and was making very slow progress. Other team members quite reasonably began adapting their code to the type checker which had come furthest. Near the end of the first eight week period L had a few days to work undisturbed by labs and lectures and completed his version of the type checker. The two-man team had been having difficulties implementing their type checker, and it was decided to use L's. Säg ngt om kommunikation, oförutsägbar utvecklingstakt i TC