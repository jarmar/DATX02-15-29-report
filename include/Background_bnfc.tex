\section{Parsing source code} \label{sec:bnfc}

\todo{Assigned David}

Similar to a normal language a programming language is built from a grammar. \todo{Is this a correct statement?}
\todo{mention that programming language grammar is (more) strict and exact}
The grammar describes ways in which the smaller pieces of the language can be put
together to form bigger ones, often with more meaning.
From this grammar a lexer is produced. The lexers job is to match all words in the
source code to tokens in the grammar. For example in english the word 'train' 
would match a noun token. 

The next step is to combine all these tokens to expressions or sentences. This
is done with a parser. To make the language as expressive as possible you build
up bigger expressions out of smaller expressions. This will create a tree
structure that will be easier to work with. 

To save a lot of work the lexer and parser can be automatically generated from
a tool when the grammar is written in Backus-Naur-form or BNF. This BNF converter, or
BNFC for short \todo{maybe glossary}, is developed at Chalmers and used in earlier courses in earlier courses meaning that the group members was already famililar with the tool (plt ref here).

\todo{Add appendix with example code and its representation}

