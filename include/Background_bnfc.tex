\section{Parsing source code} \label{sec:bnfc}

\todo{Assigned David}

%\todo{proposed change in comment / use whole or any parts you like}

A programming language is similar to a natural language in that it is described by a grammar. Programming languages are however different in that they are built from their grammar and are thus unambiguous. % and not the other way around?
The grammar describes ways in which the smaller pieces of the language can be put together to form bigger ones, often with more meaning. 

From this grammar a lexer is produced. The lexers job is to match all words in the source code to tokens in the grammar. For example in english the word 'boat' would match a noun token. This list of tokens is combined with a parser to make expressions or sentences. To make the language as expressive as possible one build up bigger expressions out of smaller expressions. This will create a tree structure that will be easier to work with. 

To save a lot of work the lexer and parser can be automatically generated from a tool when the grammar is written in Backus-Naur-form or BNF. This \gls{bnfc} \cite{bnfc}, or BNFC for short, is developed at Chalmers and used in earlier courses meaning that the group members was already famililar with the tool.



%%%% TO BE REMOVED LATER. Old texts:
% Similar to a spoken language a programming language is built from a grammar, but different from a spoken language a programming language's grammar is exact. This comes from that a spoken language's grammar is a construction in retrospect and also that it does not always follow the rules.