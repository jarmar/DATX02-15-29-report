\section{Social and economic impact}

One of the main benefits of exposing Erlang functionality to more users, by adding clean syntax and type safety, is increased productivity. The Erlang environment is good for quickly producing functioning concurrent code and less time implementing means more time using and testing. A clean and readable syntax is also more understandable which could cut down time spent on reading code, which is a big part of programming.
 
A strong type system gives another increase in productivity. The cost of fixing bugs grow significantly for each step in the life of the software and finding any bugs early on is beneficial.

Through increased parallelism there might also be positive effects on energy use. More parallelism could see decreased clock speeds and less energy spent in total.