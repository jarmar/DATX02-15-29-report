\section{Programming language}

\todo{restrict use of "the group" in the text?}

\todo{Assigned Jakob}

For implementing the compiler, Haskell was selected as the main programming language. There were several reasons for this choice. Since the aim of the project was to implement a Haskell-like language on the BEAM VM, Haskell and Erlang were prime candidates, as those choices would allow us to make use of their similarities to Hopper. 

Haskell has the advantage of being syntactically closer to what we wanted Hopper to be and being good for describing high level abstractions. Erlang, on the other hand, is closer to what would be sent to the Erlang VM, but is less intuitive for the group members to use when designing the compiler.

\todo{Improve pros/cons of Haskell and Erlang}
\todo{Why haskell is good for compiler making}%bnfc

After some consideration the group chose to implement the compiler in Haskell. One major factor was the groups familiarity with the language, some group members even having written compilers in Haskell, and its high level abstractions make it easy to work with when designing a language. Access to tools like BNFC was another plus.

Sketching out the project there were plans of possibly moving from a Hopper compiler written in Haskell to a Hopper compiler written in Hopper. This is sometimes called a self-hosting compiler since it would then compile itself. However, this was quickly considered to be out of scope for the project.
