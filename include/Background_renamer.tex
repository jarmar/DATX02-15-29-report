\section{Renaming and simplifying}

\todo{Assigned DAvid}

While BNFC is a great tool to create a parse tree from raw text the resulting 
tree structure is very verbose with a lot of constructors. To simplify this a
renamer step is introduced in the pipeline. This step converts the parse tree
to our minimally designed \acrlong{ast} or \acrshort{ast} for short. 

\begin{table}
\begin{lstlisting} 
f = if a                      f = case a of
      then b        =>              True  -> b
      else c                        False -> c
\end{lstlisting}
\caption{Simple transformation to a more general form}
\label{lst:renamer1}
\end{table}

In this step a few grammar rules could be translated to more general expressions
to simplify the AST and lessen the cases the typechecker need to cover, see
table \ref{lst:renamer1}.

The other part of the renamer module is to annotate the functions to where they
where exported from...
