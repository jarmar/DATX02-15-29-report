\section{Renaming and simplifying}

\todo{Assigned David}

While BNFC is a great tool for creating a parse tree from raw text the resulting tree structure is verbose and has a lot of constructors. To simplify this representation a renamer step is introduced in the pipeline. This step converts the parse tree to our minimally designed \acrlong{ast} or \acrshort{ast} for short. 

\begin{figure}
\begin{lstlisting} 
f = if a                      f = case a of
      then b        =>              True  -> b
      else c                        False -> c
\end{lstlisting}
\caption{Simple transformation to a more general form}
\label{lst:renamer1}
\end{figure}

In this step a few grammar rules could be translated to more general expressions to simplify the AST and reduce the number of cases the typechecker need to cover, see figure \ref{lst:renamer1}.

The Renamer has a few more task to take care of. First it gathers functions defined with patternmatching and merge them. Second it annotates the functions with the module name. This removes the ambiguity betweeen functions with the same name but defined in different modules. Lastly it checks a list of \acrlong{bif}s or \acrshort{bif}s for short to switch out them to calls to the virtual machine instead of regular user defined function calls.

\todo{BIFs, Collect definitions, qualify}
