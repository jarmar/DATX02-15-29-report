\chapter{Introduction}

% Purpose: what is the difficult stuff, list problems we will face etc. (What will we need to do? short "trailer" of what will be talked about in design & impl etc) frågeställning

% Problemformuleringen visar området inom vilket man tänker gräva, medan  problemavgränsningen  visar vilken specifik del av området man tänker gräva ut. Syftet däremot visar hur djupt man tänker gräva (mer om "djupet" kommer nedan), i betydelsen vad man vill åstadkomma,men också vem som ska kunna använda utredningen och till vad. 

problem statement
% area of research. languages/compilers?

scope
% prototype

purpose - http://en.wikipedia.org/wiki/Porpoise
% what is the difficult stuff, list problems we will face etc.
% who will use the results of the thesis? for what will they use it?

% design language, implement compiler
% what will be talked about in design & impl




\todo{Assigned to: Björn}
\todo{Comments from fackspråk}
\todo{clearer purpose, get to the core earlier}

To take advantage of both BEAM's great concurrency model with the safety of 
Haskell's typesystem this project is intended to design and implement a new 
language named Hopper.  
