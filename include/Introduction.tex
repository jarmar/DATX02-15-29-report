\chapter{Introduction}

\todo{Assigned to: Björn}

% Problemformuleringen visar området inom vilket man tänker gräva, medan  problemavgränsningen  visar vilken specifik del av området man tänker gräva ut. Syftet däremot visar hur djupt man tänker gräva (mer om "djupet" kommer nedan), i betydelsen vad man vill åstadkomma,men också vem som ska kunna använda utredningen och till vad. 

\section{Problem statement}

There is a trend toward more type checking and more parallellism in modern programming languages. Type checking to manage correctness in ever more complex systems and parallellism to utilize the growing number of cores available in computers. There currently exist eminent languages in each of these categories. We will take on the challenge of combining the merits of these fields in a new programming language called Hopper endowed with an expressive type system and with native facilities for parallell programming. The question that the project aims to answer is whether a prototype for this new language can be implemented in such a way that it is powerful, containing the desired features, and usable.

\section{Scope}

As programming language design and implementation is an intricate field of study we will need to limit the scope of the project to fit the limitations of a bachelors thesis. \todo{Prototype, Haskell famous for types therefore Haskell-like type system and syntax, Erlang famous for parallellism therefore Erlang VM.
To take advantage of both BEAM's great concurrency model with the safety of 
Haskell's typesystem this project is intended to design and implement a new 
language named Hopper.}

\section{Purpose}

\todo{What we will need to do and problems we might face in design}

\todo{What we will need to do and problems we might face in implementation}

We think this report might interest the programming language community and serve as a springboard into a more exhaustive attempt at integrating type systems with parallell capabillities.
