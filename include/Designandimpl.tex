\chapter{Design and Implementation}

\todo{Assigned Johan L}

This chapter describes the design and implementation of the Hopper language and its compiler with all the compiler components. There will be overviews of each part describing the choices made and the reasoning behind those choices.

When first designing Hopper there was already an outline of what the language should aim to be. The type safety and clean syntax of Haskell combined with the powerful framework for concurrency and parallelism supplied by the Erlang virtual machine was the key features from which Hopper would be built. Within those confines there were further choices as to what features to include, and not to include, to make the language relevant and useful.

\todo{Need help figuring out what features that are relevant to mention here!}

Write about the most relevant features, not covered in a compiler phase, here...

\section{The compiler pipeline}

% PIPELINE FIGURE

\begin{figure}[h!]
\centering
  \includegraphics[width=0.6\pdfpagewidth]{figure/pipeline}
  \caption{Hopper's pipeline}
  \label{fig:pipeline}
\end{figure}

% HOW IS HOPPER BUILT

As seen in figure~\ref{fig:pipeline} the Hopper language has five steps in its 
pipeline before it turns it over to the Erlang compiler. Each step work on the 
source code representation in a uniqe way. The steps can be summerized as
follows. The lexer and parser reads in text files and converts them to an 
abstract representation, the renamer simplify this representation for that 
typechecker whos job it is to control the types of the functions. Second to last 
there is another simplifying step before the code generator which produces Core
Erlang code. The Erlang compiler is then invoked, using the produced Core Erlang
code, and the result is Erlang assembly code that is run on the BEAM VM.

\section{Parser}

The parser is implemented by a BNF grammar that is converted to a lexer and 
parser with the BNFC tool. 

\subsection{BNFC - automatic generation}

BNFC, as described in the background, is the Backus Naur Form Converter developed at Chalmers which the group used to automatically generate a lexer and parser. The group had to write a grammar for the Hopper language in Backus Naur Form and feed it to BNFC which then, for Haskell, uses Alex and Happy to produce the lexer and parser.

\subsection{The grammar}

Writing the grammar you, in effect, describe the ways in which programs accepted in the language can be written. Hence it takes great consideration to detail to ensure that the grammar allows everything that the language is supposed to be able to express but disallows everything else.

\todo{Improve choice of words/phrasings} 

\subsection{Planning to use Alex and Happy}

Starting the project there were discussions in the group on whether or not to use BNFC to generate the lexer and parser. Choosing to get an easy start with the generated lexer and parser was accompanied by the notion that the group might want to switch to using Alex and Happy directly at a later stage. This did not come to happen during the project but might still be something to consider for the future. Implementing the groups own lexer and parser would mean greater control over the process.

\todo{does this discussion belong in background instead? fill out}


\section{Renamer}

While BNFC is a great tool for creating a parse tree from raw text the resulting tree structure is verbose and has a lot of constructors. To simplify this representation a renamer step is introduced in the pipeline. This step converts the parse tree to the language's minimally designed \gls{ast}. 

The renamer is implemented as a traversal over the parse tree obtained from the parser. In this traversal the renamer builds up a new tree of constructors for the \gls{ast}. A few grammar rules are translated to more general expressions to simplify the \gls{ast} and reduce the number of cases the type checker and code generator need to cover, see figure \ref{lst:renamer1}.

\begin{figure}[ht]
\centering
\lstset{mathescape=true,literate={=>}{$\Rightarrow{}$}{1}}%,frame=lrtb}
\begin{minipage}[t]{0.30\linewidth}
    \centering
    \begin{lstlisting}
f = if a
      then b
      else c
    \end{lstlisting}
\end{minipage}
\begin{minipage}[t]{0.09\linewidth}
    \centering
    \begin{lstlisting}

=>
    \end{lstlisting}
\end{minipage}
\begin{minipage}[t]{0.30\linewidth}
    \centering
    \begin{lstlisting}
f = case a of
      True  -> b
      False -> c
    \end{lstlisting}
\end{minipage}
\caption{Simple transformation to a more general form}
\label{lst:renamer1}
\end{figure}

The Renamer has a few more tasks to take care of. First it transforms the \glspl{adt} and gathers functions defined with pattern matching and merges them. Second it annotates all the identifiers, except variables bound in a pattern matching scope, with the module name. This removes the ambiguity between functions that has the same name but are defined in different modules. Lastly it checks a list of \glspl{bif} to switch them out for calls to the virtual machine instead of regular user defined function calls.

While implementing the renamer much consideration was take to what later stages of the compilation needed. For example, while Erlang allows pattern matching function Core Erlang does not. This is the reason for merging function definitions to a single definition with a case expression and each previous function as a clause. For the same reason the type data type was changed to fit better with the type checker.

The renamer also processes the internal representations of \glspl{adt}. When the renamer encounters an \gls{adt} definition it just converts each constructor to a function with just a type. With this the type checker can verify that it is used correctly and leaving it fully to the code generation to implement it without any dependencies. 

\todo{final sentence uses "this", "it", "it", "it" but is referencing different stuff in some of them, make sure that it is always clear what "it" is (might not need to change all of them)}

\section{Dependency checker}

\todo{Assigned Jakob}
The dependency model of Hopper is rather simple. Modules declare dependency
on other modules using \texttt{import} statements. Hierarchical module
names are supported, e.g. \texttt{A.B.C}. Only a single
source directory is supported, so there is a straight-forward map from module
names to file paths: the above example maps to \texttt{A/B/C.hpr}. This
restriction simplifies 
\todo{interface files}

\todo{topological sort}


\section{Type checker}

\todo{Assigned to Liam}

\subsection{The type system}
Rough draft:
\

What do we do in TC/Convert?\

•	Implement constructors (thus allowing codegen to have no knowledge of types)
•	Give some functions implicit type arguments : abused syntax to achieve this. In some ways more powerful than Haskell’s class system (because it is done at runtime in normal Hopper), but more boilerplatey and performance intensive. Allowing values to vary based on the their type (i.e. ad-hoc-polymorphism, where polymorphic values are more than just a black box, see Theorems for Free [reference?]) is essential for implementing typesafe message passing, because type information must be sent in messages at runtime to distinguish between values with same runtime representation but different types (e.g. an empty list of Booleans and an empty list of numbers).
•	Implemented receive: abused syntax using case receive of ... -> ...
•	A module that Converts to/from AST.AST. This is bad and is purely technical debt. Why it was created and maintained (two different ASTs, inertia, poor coordination) can be elaborated on in Discussion.
•	Prim.apply\

What can we do with the type system?\

•	Higher order functions, a lot of that Haskelly goodness.
•	IO, other types which describe processes and the connections between them. Because the BEAM is a high level VM designed for a dynamic programming language, coercing between types in Hopper does not have the same catastrophic consequences as in Haskell (which compiles to native machine code: coercing something on the heap can lead to a fatal crash e.g. because you treat a number as a pointer). That means it is rather comfortable to treat the same value as several different types in Hopper: for example a Number may be both a Natural and a Prime. We do not yet know where that leads, but it seems exciting. 
\
(Here ends the draft I added tonight)
\

\\
Hopper uses the Damas-Hindley-Milner (HM) type system for the lambda calculus. HM has parametric polymorphism enabling (wording, what do I mean and what does this say???) type inference of type schemes - types with generic, quantified type variables. Further HM has the Algorithm W which not only infers the most general type for expressions but also has existing soundess and completeness proofs. These proofs means that for our well formed expressions we will always find a type and that it will be the correct type.

\subsection{Preprocessing}

Damas-Hindley-Milner is a type system for the lambda calculus and the parsed Hopper code has a different representation. The parsed code is transformed into an AST and the expressions written in Hopper is turned into semantically equivalent expressions in the lambda calculus. Hopper type inference is done on a simple language which is lambda calculus extended with let expressions and the fixpoint combinator. Lambda calculus in itself is Church-Turing complete and could thus represent these added constructs but they simplify ... (add stuff here)

\begin{description}
\item[The simple language] \hfill
\begin{description}
  \item[Variables] \hfill \\
    A variable is an expression.\\
    Ex. x is a valid expression.
   \item[Abstraction] \hfill \\
    An abstraction of a variable over some expression. (wording?)\\
    Ex. (\textbackslash x . x) is a valid expression.
   \item[Application] \hfill \\
    An application of an expression to another expression is an expression.\\
    Ex. (f x) is a valid expression.
   \item[Extensions] \hfill
\begin{description}
    \item[Let] \hfill \\
      A let expression, defining a variable within the scope\\
      of an expression, is an expression.\\
      Ex. (let x = n in (f x)) is a valid expression. 
    \item[Fix] \hfill \\
      The fixpoint combinator enables recursive expressions.\\
      A fixpoint combinator y satisfies y f = f (y f).
\end{description}
\end{description}
\end{description}

\subsection{Type inference with Algorithm W}

Where the original Algorithm W uses a top-down approach we have chosen to implement it using a bottom-up approach (sometimes called Algorithm M REF: generalizing hindley-milner type inference algorithms, Heeren,Hage,Swierstra ). The bottom-up approach generates constraints on the types in a first pass and then solves these constraints producing the most general type for each expression.

\begin{description}
  \item[Generating constraints by inference rules] \hfill \\
A first pass generates constraints on the types by a set of inference rules (see appendix). \todo{FILL OUT}
  \item[Solving constraints by solving rules] \hfill \\
Secondly we solve the inferred types using the accumulated constraints to produce the most general type for each expression. \todo{FILL OUT}
\end{description}

\section{Code generator}

\todo{Assignee Liam}

Codegen 
We chose to represent fully applied constructors as tuples with the constructor name in the first position, a common pattern in Erlang used to simulate ADTs.  While not always optimal from a performance perspective (e.g. lists would be more efficiently represented using Erlang’s primitive list data types, Just x  {x} is more efficient than Just x  {just,x}), it makes the compiler simpler; the returned values easier to read, and the code more interoperable with idiomatic Erlang.
Functions: 
In erlang every function has an arity, the number of arguments it takes. Applying a different number of arguments to it will result in a runtime exception. Hopper’s functions are modelled after Haskell’s, in that they are all of arity 1. Haskell’s (and therefore Hopper’s) clever use of whitespace to indicate function application makes partial application of functions much less verbose than in Erlang, where an anonymous function needs to be created to simulate partial application. 
Example: the result of multiplying every element in a list by 3 in Hopper versus in Erlang
•	Hopper: map (mul 3) list,
•	Erlang:  lists:map(fun(X)-> X + 1 end,List)
(... continued explanation about why arities of functions can't be predicted and therefore why currying all functions isn't as expensive as it seems (because it has to be done in higher order functions like map anyway)

Generally the last step of a compiler is the code generation step. The code generator
component should take an \gls{ast} much like one of the target language and translate it
to actual code strings of the target language. Additionally, this component may also
perform some optimizations on the generated code before returning it.

\subsection{Implementation}

The code generator is implemented using the Language.CoreErlang \cite{CoreErlang} package
from Hackage (maybe also bibliography link, or glossory). This package contains a Core Erlang
parser, a data structure used for representing Core Erlang \gls{ast}s and a pretty printer for
said data structure. The code generator makes use of this package by building Core Erlang
\gls{ast}s and then simply invoking the pretty printer on them to  produce Core Erlang.

\subsection{Function applications}
\todo{Include nice example figure! Liam knows what kind of function will be a good example}
%Good example: map (add 1) xs

A substantial difference between Haskell and Erlang is how functions are applied to arguments. In Haskell, all multi parameter functions may be fully or partially applied. When a function is fully applied, it is given its maximum number of arguments, and it returns a value of a single type. However, when a function is partially applied, i.e. it is given any number of arguments from 1 to (but not including) its maximum number of arguments, it instead returns a function of the types that were excluded from the partial application.

In Erlang, a function is identified by its name and its arity, and can only be applied to a number of arguments that is exactly that of its arity. However, the types of the parameters passed are not significant when the function is called (though they may be of significance when the function body is evaluated), and function names can be shared by different functions if they have different arities.

Hopper attempts to preserve Haskell's support for partial application. This, in combination with the fact that the Erlang compiler requires all function definitions to have a determined arity at compile time, results in the translation of functions being one of the non-trivial tasks of the code generator.

\todo{Present and explain the curry solution}

\subsection{Algebraic data types and constructors}

Another area where Haskell and Erlang differ is the use of \gls{adt}'s. \gls{adt}'s are types 
composed by other types, and their values are analyzed using pattern matching. The values of are created using constructors, a construct which may or may not take a number
of parameters for creating the value and then returns it. When pattern matching over \gls{adt}'s, their
constructors are typically used for identifying and handling specific instances of the data types.

These kind of data types do not exist in Erlang in the same way as they do in Haskell, i.e. with their heavy reliance
on constructors. When using pattern matching in Erlang, an entity called atom is instead used for
handling specific cases. An atom is a literal constant with a name. Variables can hold atoms, and said
variables can be pattern matched over different atoms for identifying different cases.

In Hopper, \gls{adt}'s are implemented and translated to Core Erlang in two specific cases: when an
they are processed, and when they are declared. Like in Haskell, variables in Erlang can
hold any value, so the only problem to solve was how the values of \gls{adt}'s should be represented in
Erlang. The solution drew inspiration from the book \textit{Implementing functional languages: a tutorial} \cite{FunTutorial},
which (among other things) describes an approach to this problem where the values of \gls{adt}'s are translated to tuples,
holding the constructors (represented by unique integer values) as a tag, and their correlated values.
\gls{adt}'s in Hopper are translated in the same manner, but translates the constructors to atoms instead,
to mimic the conventional structure of messages in Erlang.
Declared constructors are simply translated to functions, that takes a number of parameters (if any), and
returns the value as a tuple representation as described above.


\begin{figure}[!htb]
\centering
\begin{lstlisting} 
data Tree = Tree Int Tree Tree | Empty

Tree 5 (Tree 3 (Tree 1 Empty Empty) 
               (Tree 4 Empty Empty)) 
       (Tree 7 Empty Empty)
\end{lstlisting}
\caption{Hopper ADT with example value}
\label{lst:hopperAdt}
\end{figure}

\begin{figure}[!htb]
\centering
\begin{lstlisting} 
{'Tree', 5, {'Tree', 3, {'Tree', 1, 'Empty', 'Empty'},
                        {'Tree', 4, 'Empty', 'Empty'}},
            {'Tree', 7, 'Empty', 'Empty'}}
\end{lstlisting}
\caption[Core Erlang translation of constructor and example value]
 {Core Erlang translation of the Hopper constructor and example value seen in figure \ref{lst:hopperAdt}}
\label{lst:coreAdt}
\end{figure}

\todo{Include example figure of constructors. Caption should tie in with fun app sub section}

\subsection{Emulating the erlang compiler}

One downside to the choice of compiling to Core Erlang and then invoking the erlc
(bibliography link here) is the loss of the additions and optimizations the erlc
compiler makes when it compiles Erlang to Core Erlang in its pipeline. To deal with this
the code generator has been given additional functionality to emulate the pre Core Erlang
behaviour of the erlc compiler. \todo{Keep anyway? Mention the non matching case exception}
