\chapter{Conclusion}

\todo{Point out the important good and bad stuff. Good: it can be done. Bad: We did it poorly}

%% A more general discussion about Hopper as a whole.

The initial goal of the project was to find out if a Haskell-like language for the Erlang VM was doable and if so - would it be useful? \todo{add/change to real purpose after introduction etc gets a better formulated problem/question}  The conclusion we can draw from our results is that it is a good idea even though the Hopper language still is not a useful product for serious software development. What was produced in the bachelors thesis project can be seen as a prototype, and given time and resources Hopper can be made into a useful programming language.

Hopper could be used today as the pure part of an application. For example, using Hopper to implement a difficult algorithm but using Erlang as the glue to the rest of the system. However, a compiler should be user friendly in the sense of giving descriptive error messages for faulty code, give warnings about inconvenient code, which Hopper doesn't. 

\todo{fill out and fix after results and introduction/purpose is done}