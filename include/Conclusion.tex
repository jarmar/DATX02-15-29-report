\chapter{Conclusion}

We set out to see if it was possible to combine an expressive type system with native facilities for concurrent and parallel programming. As a model for our type system and language syntax we chose Haskell and for the concurrency and parallelism primitives we chose Erlang. Both are very competent languages in their respective area. From the results and discussion chapters it is clear that it is possible to implement a type system for a functional language that runs on the Erlang VM. It is however still not clear how to utilize the concurrent and parallel features of Erlang in this typed environment.

It is evident from the discussion that most of the shortcomings of the project originates from administrative errors and a lack of communication coupled with a tight time frame. A lot of time that could have been put into design, experimental work and implementation was instead put into integrating contradicting components, reworking features and converging of ideas. The time that was actually spent on feature development and experimentation proved fruitful. 

The developed product, though somewhat limited, does what it is intended to do rather well. Its syntax is expressive and simple, it has a strong type system, and executing generated code produces the correct result for the examples we have written. There were a number of features designed and planned that did not make it into the final product, for example full support for message passing between processes written in Hopper and rich informative error output from the compiler. This adds to our belief that Hopper has potential. 

We conclude that the future of Hopper is promising. Given time and resources, the programming language Hopper has the potential of growing into a useful alternative when developing applications for the Erlang virtual machine.
