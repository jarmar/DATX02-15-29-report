\chapter{Conclusion}

\todo{Assigned Johan L temporarily}

\todo{Conclusive text that closes the report}

%Here I think we should start with something like "The initial purpose of the program was to investigate if a Haskell like   language for the Erlang VM is a good idea. The conclusion drawn from this project is that it is a good idea, although a 15hp project is not enough to produce a useful product for serious software development. However, the result of the project can be seen as a prototype, and given time and resources, Hopper can be made into a useful programming language". Just think about the content here, not the phrasing. I dont know if the method should be analyzed at all, the purpose ofthe project was to investigate the posibilites of Hopper, not to see how well our weird agile aproach works in a project... Maybe add this as a question for fackspråk 2? // Johan WS

%Agree, I think we should focus on if we succeeded in creating something useful, or if it is at least possible that Hopper could be useful given time and work as you say. If we get a good question to be answered included in introduction or something we can relate to that. / JL

%% A more general discussion about Hopper as a whole.

\todo{Almost a summary of the discussion, did we answer the question stated in introduction/abstract/wherever??}

\todo{Keep it short and to the point, did we do what we set out to do etc, compare results to goals}

\todo{Was it a good idea? Is it a usable product? If not, can it be usable given more time? Relate to stated purpose}

\todo{Was the method of choice good? What could have been improved?}

\todo{The impact of non expressive goals (The goal was to implement a language and see how far we could go)}

The initial goal of the project was to find out if a Haskell-like language for the Erlang VM was doable and if so - would it be useful? \todo{add/change to real purpose after introduction etc gets a better formulated problem/question}  The conclusion we can draw from our results is that it is a good idea even though the Hopper language still is not a useful product for serious software development. What was produced in the bachelors thesis project can be seen as a prototype and given time and resources Hopper can be made into a useful programming language.

\todo{fill out}