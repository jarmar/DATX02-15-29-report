\section{Related work}
%nytt
Previous attempts to combine the features of Erlang and Haskell have been made, illustrating the demand for a combination of the languages' features.

Attempts to port Erlang features to Haskell include Cloud Haskell \cite{cloudhaskell}, which is ``Erlang in a library'', essentially implementing Erlang's runtime system in Haskell. However, a lack of syntactic support for message passing in Haskell make shipping expressions between nodes more verbose than in Erlang.

Efforts have also been made to port Haskell features to Erlang. One example is the Haskell to Erlang transpiler backend to YHC \cite{yhc}, which preserves laziness. A similar project is Haskerl \cite{haskerl}, which compiles a subset of Haskell to the \gls{beam}. However, we did not find projects which also attempted to integrate \gls{beam}'s concurrency capabilities on a language level, which is one of Hopper's goals.