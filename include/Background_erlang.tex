\section{Core Erlang and BEAM}

Core Erlang \cite{CoreErlangIntro} is an intermediate language in the Erlang compilation suite, mainly used for simplifiying operation on Erlang source code. Such operations can include (but not limited to) parsing, compiling and debugging. Some of the main purposes of Core Erlang are to be as regular as possible, provide clear and simple semantics. The language is since release 8, released in 2001, an official part of the OTP/Open Source Erlang Distribution.

BEAM is the name of the Erlang virtual machine. Erlang and Core Erlang source code may be compiled to .beam files, which contain instructions from the BEAM instruction set. These instructions are interpreted by the virtual machine at run time.