\section{The Compiler}

The Hopper compiler takes a list of \texttt{.hpr} files and compiles them, and the modules they depend on if needed. Apart from this, it may also take a selection of command line options for producing different kinds of helpful output. Optional command line arguments and explanations of their functionality can be seen in table \ref{tab:flags}.

\begin{table}[!htb]
\centering
\begin{tabular}{| l | l | l |}
\hline
Shorthand flag & Full flag            & Functionality\\
\hline
\texttt{-v}    & \texttt{--verbose}   & Print info about compilation progress\\
\texttt{-p}    & \texttt{--parse}     & Write the parse tree to a \texttt{.parse.hpr} file\\
\texttt{-a}    & \texttt{--ast}       & Write the abstract syntax tree to a \texttt{.ast.hs} file\\
\texttt{-t}    & \texttt{--typecheck} & Write the type checked tree to a \texttt{.typed.hs} file\\
\texttt{-c}    & \texttt{--core}      & Write the generated Core Erlang code to a \texttt{.core} file\\
\texttt{-b}    & \texttt{--nobeam}    & Skip writing the target \texttt{.beam} file\\
\hline
\end{tabular}
\caption[Optional arguments for the Hopper compiler]{Optional arguments for the Hopper compiler}
\label{tab:flags}
\end{table}



\subsection{Modules}

\begin{figure}[!htb]
\centering
\begin{minipage}[b]{0.43\linewidth}
\centering
\begin{lstlisting}
module Hop (f, g, h) where

import Tree.BinaryTree
import MyModel

...
\end{lstlisting}
\end{minipage}
\caption{Export listing and import statements}
\label{lst:exportimport}
\end{figure}

Hopper supports a module system by providing export listings for declaring interfaces and import statements for declaring dependencies. Cyclical dependencies are reported as errors for simplicity and as a way to enforce separation of concerns.
Information about the modules are stored in dedicated interface files. Type information propagation is however broken in the current version. In figure \ref{lst:exportimport} the syntax of exports and imports are shown, where \texttt{f}, \texttt{g} and \texttt{h} are exported functions.

The fact that the Hopper compiler generates \gls{beam} modules allows for a degree of interoperability: Hopper modules may be integrated in applications written in Erlang in a transparent way. Some parts of the compiler are prepared for enabling Hopper code to make use of modules written using Erlang.

\subsection{Type checker}

The type checker is able to diagnose and report type errors in Hopper code as well as infer the types of definitions.
It provides support for polymorphic \glspl{adt} such as lists and \texttt{Maybe} and polymorphic functions such as \texttt{map} and \texttt{fold}. %In addition, it supports type safe receive clauses and ad-hoc polymorphism in an experimental form.