\section{Code generator}

Generally, the last step of a compiler is the code generation step. The code generator
component should take a collection of \glspl{ast} and translate them to the target language. Additionally, this component may perform some optimizations on the generated code before it finishes.

\subsection{Implementation}

The code generator is implemented using the \texttt{Language.CoreErlang}\cite{CoreErlang} package. It contains a Core Erlang parser, a set of data types used for representing Core Erlang \glspl{ast} and a pretty printer for said data types. The code generator makes use of this package by building Core Erlang \glspl{ast} and then invoking the pretty printer on them to produce Core Erlang source code.

\subsection{Function applications}

A substantial difference between Haskell and Erlang is how functions are applied to arguments. In Haskell, all functions are of \gls{arity} 1, and when functions are applied to several arguments, they are \glslink{currying}{curried} during runtime. This means that functions can be partially applied, in which case they return a function. This is useful when for example there is a need for composing new functions from already defined ones in a terse manner.

Hopper attempts to mimic Haskell's support for partial application. This, in combination with the fact that the Erlang compiler requires all function definitions to have a determined \gls{arity} at compile time, results in the translation of functions being one of the non-trivial tasks of the code generator.

This problem was solved by defining an Erlang function named \texttt{curry}, which turns all expressions either into functions of \gls{arity} 1 or into values. The Core Erlang code produced by Hopper will contain calls to this function, resulting in functions being curried during runtime. The implementation of the curry function can be seen in Appendix~\ref{app:curry}.

In figure \ref{lst:hopperFun} and figure \ref{lst:coreFun} a simple Hopper function and the resulting Core Erlang code are shown respectively.

\begin{figure}[!htb]
\centering
\begin{minipage}[b]{0.38\linewidth}
\centering
\begin{lstlisting}
f :: Int -> String
f 0 = "False"
f 1 = "True"
\end{lstlisting}
\end{minipage}
\caption{Simple Hopper function}
\label{lst:hopperFun}
\end{figure}

\begin{figure}[!htb]
\centering
\begin{lstlisting}
'f'/1 =
  fun(X@1) ->
    case {X@1} of
      {P_arg1} when 'true' ->
        case {P_arg1} of
          {0} when 'true' ->
            "False"
          {1} when 'true' ->
            "True"
          ({_cor1} when 'true' ->
             primop 'match_fail'({'case_clause', _cor1})
           -| ['compiler_generated'])
        end
      ({_cor1} when 'true' ->
         primop 'match_fail'({'case_clause', _cor1})
       -| ['compiler_generated'])
    end
'__f'/0 =
  fun() ->
    apply call 'erlang':'make_fun'
               ('Prim', 'curry', 1)('f'/1)
\end{lstlisting}
\caption[Generated Core Erlang translation of function]
 {Core Erlang translation of the Hopper function seen in figure \ref{lst:hopperFun}}
\label{lst:coreFun}
\end{figure}

\subsection{Emulating the Erlang compiler}

One downside to the choice of compiling to Core Erlang and then invoking the Erlang compiler \todo{bibliography link here} is the loss of additions and optimizations. To deal with this the code generator has been given additional functionality to emulate the behavior of the Erlang compiler prior to generating Core Erlang. 

The currently implemented behavior is the generation of \texttt{module\_info} functions that return information about the module when invoked, and an additional case to all case expressions which catches any unmatched case and throws a run time exception.
