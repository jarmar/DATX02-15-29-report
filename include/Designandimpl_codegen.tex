\section{Code generator}

\todo{Assignee JohanWS/Liam}

Generally the last step of a compiler is the code generation step. The code generator
component should take an \gls{ast} much like one of the target language and translate it
to actual code strings of the target language. Additionally, this component may also
perform some optimizations on the generated code before returning it.

\subsection{Implementation}

The code generator is implemented using the Language.CoreErlang \cite{CoreErlang} package
from Hackage (maybe also bibliography link, or glossory). This package contains a Core Erlang
parser, a data structure used for representing Core Erlang \gls{ast}s and a pretty printer for
said data structure. The code generator makes use of this package by building Core Erlang
\gls{ast}s and then simply invoking the pretty printer on them to  produce Core Erlang.

Due to the differences between Hopper and Core Erlang, and the additional type constraints
of the Hopper language, the type checker and the code generator require differently
structured \gls{ast}s. To deal with this another \gls{ast} transformation step has been added
to the compilation time line before the code generation. This step attempts to seperate variables
and function identifiers from each other, and bind function calls with the tokens that
should be used as parameters in said function calls.

\todo{Subsection about constructor to atom translations}

\todo{Subsection about function applications and the built in curry function}

\subsection{Emulating the erlang compiler}

One downside to the choice of compiling to Core Erlang and then invoking the erlc
(bibliography link here) is the loss of the additions and optimizations the erlc
compiler makes when it compiles Erlang to Core Erlang in its pipeline. To deal with this
the code generator has been given additional functionality to emulate the pre Core Erlang
behaviour of the erlc compiler. (Proceed to list features here)