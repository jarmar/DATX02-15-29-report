\chapter{Background}

% WHAT IS A COMPILER 

A 

% HOW IS A COMPILER BUILT 

When building a compiler it can be resembled to a pipe where text flows through 
while transformed into different representations and in the end outputted
to runnable code to either run on the computer itself or on a virtual machine
that emulates a computer.

Now is Hopper really a transpiler or a translating compiler. Instead of 
outputting runnable code it outputs other code that is interpreted and compiled
by the Erlang compiler, erlc. With this the project could very much be 
simplified (ref to read more in discussion).

% PIPELINE FIGURE

\begin{figure}[h!]
\label{fig:pipeline}
\centering
  \includegraphics[width=0.6\pdfpagewidth]{figure/pipeline}}
\end{figure}

% HOW IS HOPPER BUILT

As seen in figure~\ref{fig:pipeline} the Hopper language has five steps in its 
pipeline before it turns it over to the Erlang compiler. Each step work on the 
source code representation in a uniqe way. The steps can be summerized as
follows. The lexer and parser reads in text files and converts them to an 
abstract representation, the renamer simplify this representation for that 
typechecker whos job it is to control the types of thefunctions. Second to last 
there is another simplifying step before the code generator who writes the
assembly code that is run on the BEAM VM.

% OTHER SECTIONS

\section{Parsing source code} \label{sec:bnfc}
Similar to a normal language a programming language is built from a grammar.
The grammar describes ways in which the smaller pieces of the language can be put
together to form bigger ones, often with more meaning.
From this grammar a lexer is produced. The lexers job is to match all words in the
source code to tokens in the grammar. For example in english the word train 
would match a noun token. 

The next step is to combine all these tokens to expressions or sentences. This
is done with a parser. To make the language as expressive as possible you build
up bigger expressions out of smaller expressions. This will create a tree
structure that will be easier to work with. 

To save a lot of work the lexer and parser can be automatically generated from
a tool when the grammar is written in Backus-Naur-form or BNF. This BNF converter, or
BNFC for short, is developed by Chalmers and used in earlier courses meaning that
the group members was already famililar with the tool (plt ref here).



\section{Renaming and simplifying}

\todo{Assigned David}

While BNFC is a great tool for creating a parse tree from raw text the resulting tree structure is verbose and has a lot of constructors. To simplify this representation a renamer step is introduced in the pipeline. This step converts the parse tree to our minimally designed \acrlong{ast} or \acrshort{ast} for short. 

\begin{figure}
\begin{lstlisting} 
f = if a                      f = case a of
      then b        =>              True  -> b
      else c                        False -> c
\end{lstlisting}
\caption{Simple transformation to a more general form}
\label{lst:renamer1}
\end{figure}

In this step a few grammar rules could be translated to more general expressions to simplify the AST and reduce the number of cases the typechecker need to cover, see figure \ref{lst:renamer1}.

The Renamer has a few more task to take care of. First it gathers functions defined with patternmatching and merge them. Second it annotates the functions with the module name. This removes the ambiguity betweeen functions with the same name but defined in different modules. Lastly it checks a list of \acrlong{bif}s or \acrshort{bif}s for short to switch out them to calls to the virtual machine instead of regular user defined function calls.

\todo{BIFs, Collect definitions, qualify}


