\section{Semantic analysis}

Background

An important design decision to make when implementing a programming language is wether to perform some semantic analyses of the code at compile time and what to include in such case. Semantic analyses can for example contribute error detection and optimized code. The cost is increased complexity of the compiler and for some kinds of analysis prohibitive compilation times.

A common and well studied category of analyses is type inference or type checking. Type systems vary in their expressiveness but essentially allow us to prove desireable properties of our code.

One of the driving motivations for the Hopper project was to bring an expressive Haskell-like type system From the outset one of the  aim of Hopper

------------------

what
  in general
    a step in the compilation process
    examine details of the code which cannot be captured be the grammar itself.
    example control flow analysis
  type inference
    a formal system
    system F in Haskell
    we use Hindley-Milner type system to begin with

why
  in general
    prove properties
    show correctness/show absence of errors
  type inference
    tractable
    well established

how
  preprocess
    the simple language
      lambda calc
        expressiveness etc
        extensions (let, fix)
    transformations
      parse -> AST ?
      AST -> simple language
      important: keep semantics
  type inference
    generating constraints by inference rules
    solving constraints by solving rules
  postprocess
    looking upp the solved types and reporting them
